\begin{abstract}
% Introduce the societal problem
Across all levels of government in the United States (U.S.), transportation and planning agencies have committed to increasing bicycle use. However, despite documented goals at the federal, state, and municipal levels, most government agencies have been unsuccessful in meeting their bicycle mode share targets.

% Link bicycle mode choice models to the main problem of increasing cycling rates
When analyzing how city and state transportation agencies can increase their bicycle commuting rates, it is clear that bicycle infrastructure projects can affect their constituents' travel behavior. However, to best allocate funds in support of their bicycling agendas, government agencies must judge the extent to which each possible project will increase bicycle usage. Such judgements are typically based on an agency's travel demand model---specifically, on the agency's mode choice model. Consequently, this dissertation focuses on improving the mode choice models used to evaluate bicycle infrastructure projects.

% Detail the broad theoretical concerns that the dissertation addresses.
In particular, this dissertation addresses three issues with current bicycle mode choice models.
\end{abstract}